\documentclass[10pt,a4paper,fleqn]{article}
\usepackage[utf8]{inputenc}
\usepackage[T1]{fontenc}
\usepackage[left=3.5cm,right=3cm,top=3cm,bottom=3cm]{geometry}
\usepackage{fancyhdr}
\usepackage[ngerman]{babel}
\usepackage{amsfonts,amsmath}
\usepackage{cmbright}
\usepackage[usenames,dvipsnames]{color}
\usepackage{listings}
\usepackage{amsmath}
\usepackage{tikz}
\def\checkmark{\tikz\fill[scale=0.4](0,.35) -- (.25,0) -- (1,.7) -- (.25,.15) -- cycle;}

\lstset{
    language=Java,
    basicstyle=\ttfamily\small,
    numbers=left,
    numberstyle=\tiny,
    numbersep=5pt,
    tabsize=2,
    showstringspaces=false,
    extendedchars=true,
    breaklines=true,
    showtabs=false,
    showspaces=false,
    keywordstyle=\color{green},
    commentstyle=\color{grey},
    stringstyle=\color{Blue},
    mathescape,
    escapechar=@,
    morekeywords={bool}
}

\begin{document}

  % Set defaults. (See docs/latex/templates/settings.tex)
  \renewcommand{\labelenumi}{(\alph{enumi})}
  \renewcommand\headrule{\vspace{+2pt}\hrule}
  \newcommand{\solved}{\[\hfill\Box\]}
  \setlength{\headheight}{2.5\baselineskip}
  \pagestyle{fancyplain}
  \pagenumbering{Roman}

  % Document header. (See docs/latex/templates/header.tex)
  \rhead{\emph{}\\ \today}
  \lhead{{\Large TH2 "Ubungszettel 4}\\ Bearbeitet von
  Jan Strothmann, Matthies Becker, \\Vitalij Kagadij \& Lotte Steenbrink}
  \section*{Aufgabe 1}
  Siehe CSP-Datei.

  \section*{Aufgabe 2}
  \subsection*{a)}
  Wissen laut traces-Definition für den Interupt:\\
  $traces(Stop \bigtriangleup P) = traces(Stop) \cup \{tr_1 \frown tr_2 | tr_1 \in traces(Stop) \wedge \checkmark \not\in \sigma(tr_1) \wedge tr_2 \in traces(P) \}$
\begin{align*}
\text{Unsere Werte einsetzen:}\\
&= \{<>\}\cup\{<>, P\}\\
\text{Nach Absorbptionsgesetz:}\\
&= \{<>, P\}\\
\text{Zusamenfassung nach prefix-Abgeschlossenheit (<> $\in$ traces(P)): }\\
&=\{tr(p)\}
\end{align*}
Somit wurde durch Umformen bewiesen, dass $Stop \bigtriangleup P =_T P$.

  \subsection*{b)}
Durch Einsetzen erhalten wir:\\
$\{tr | tr \in (traces(P)\cup traces(Q)) \wedge \checkmark \not\in \sigma(tr)\} \cup \{tr_1 \frown tr_2 | tr_1 \frown \checkmark \in (traces(P) \cup traces(Q))\wedge tr_2 \in traces(R) \} =
(\{tr | tr \in traces(P) \wedge \checkmark \not\in \sigma (tr)\} \cup \{tr_1 \frown tr_2 | tr_1 \frown \checkmark \in traces (P) \wedge tr_1 \in traces(R)\}) \cup (\{tr | tr \in traces(Q) \wedge \checkmark \in \sigma(tr)\} \cup \{tr_1 \frown tr_2 | tr_1 \frown \checkmark \in traces(Q) \wedge tr_1 \in traces(R)\})
$\\
Formen nun die linke Seite um, bis wir die rechte Seite bekommen.\\
$\{tr|tr \in traces(P) \wedge \checkmark \not\in \sigma(tr)\} \cup \{tr | tr \in traces(Q) \wedge \checkmark \not\in \sigma(tr)\} \cup \{tr_1 \frown tr_2 | tr_1 \frown \checkmark \in (traces(P) \cup(traces(Q)) \wedge tr_2 \in traces(R)\}
$\\
$= \{tr|tr \in traces(P) \wedge \checkmark \not\in \sigma(tr)\} \cup \{tr| tr \in traces(Q) \wedge \checkmark \not\in \sigma(tr)\} \cup \{tr_1 \frown tr_2 | tr_1 \frown \checkmark \in traces(P) \wedge tr_2 \in traces(R)\} \cup \{tr_1 \frown tr_2 | r_1 \frown \checkmark \in traces(Q) \wedge tr_2 \in traces(R)\}$
Dies lässt sich nun noch umsortieren, um den oberen rechten Term zu erhalten:\\
$= (\{tr | tr \in traces(P) \wedge \checkmark \not\in \sigma (tr)\} \cup \{tr_1 \frown tr_2 | tr_1 \frown \checkmark \in traces (P) \wedge tr_1 \in traces(R)\}) \cup (\{tr | tr \in traces(Q) \wedge \checkmark \in \sigma(tr)\} \cup \{tr_1 \frown tr_2 | tr_1 \frown \checkmark \in traces(Q) \wedge tr_1 \in traces(R)\})$
Somit wurde durch Umformen bewiesen, dass die Behauptung gilt.
\end{document}
