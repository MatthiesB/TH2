\documentclass[10pt,a4paper,fleqn]{article}
\usepackage[utf8]{inputenc}
\usepackage[T1]{fontenc}
\usepackage[left=3.5cm,right=3cm,top=3cm,bottom=3cm]{geometry}
\usepackage{fancyhdr}
\usepackage[ngerman]{babel}
\usepackage{amsfonts,amsmath}
\usepackage{cmbright}
\usepackage[usenames,dvipsnames]{color}
\usepackage{listings}
\usepackage{amsmath}
\usepackage{tikz}
\usepackage{enumitem}

\def\checkmark{\tikz\fill[scale=0.4](0,.35) -- (.25,0) -- (1,.7) -- (.25,.15) -- cycle;}

\lstset{
    language=Java,
    basicstyle=\ttfamily\small,
    numbers=left,
    numberstyle=\tiny,
    numbersep=5pt,
    tabsize=2,
    showstringspaces=false,
    extendedchars=true,
    breaklines=true,
    showtabs=false,
    showspaces=false,
    keywordstyle=\color{green},
    commentstyle=\color{grey},
    stringstyle=\color{Blue},
    mathescape,
    escapechar=@,
    morekeywords={bool}
}

\begin{document}

  % Set defaults. (See docs/latex/templates/settings.tex)
  \renewcommand{\labelenumi}{(\alph{enumi})}
  \renewcommand\headrule{\vspace{+2pt}\hrule}
  \newcommand{\solved}{\[\hfill\Box\]}
  \setlength{\headheight}{2.5\baselineskip}
  \pagestyle{fancyplain}
  \pagenumbering{Roman}

  % Document header. (See docs/latex/templates/header.tex)
  \rhead{\emph{}\\ \today}
  \lhead{{\Large TH2 Theorie 4}\\ Bearbeitet von
  Jan Strothmann, Matthies Becker, \\Vitalij Kagadij \& Lotte Steenbrink}

\section{CSP Semantik}
\subsection{Unterschied}

\begin{equation*}
\begin{split}
failures(P1) = \{&(<>,\{\}),\\
                 &(<a>, \{a\}),\\
                 &(<a,b>, \{a,b,c, \checkmark\}),\\
                 &(<a,c>,\{a,b,c\}),\\ &(<a,c,\checkmark>, \{a,b,c, \checkmark\})\}\\
failures(P2) = \{&(<>,\{c\}),\\
                 &(<>, \{b, \checkmark\}),\\ &(<a>, \{a,c\}),\\
                 &(<a>, \{a,b,\checkmark\}),\\
                 &(<a,b>, \{a,b,c\}), \\&(<a,c>, \{a,c,b,\checkmark\}),\\ &(<a,b,\checkmark>, \{a,b,c,\checkmark\})\}
\end{split}
\end{equation*}

\subsection{Laws widerlegen}
TODO
\subsection{Laws anwenden}
TODO

\section{Refinement}
\subsection{Nachweis von Automaten-Äquivalenz}
\begin{lstlisting}
channel a, b

N0 = SKIP |~| (a -> N0 [] b -> N1 [] b -> N2)
N1 = (a -> N3 [] b -> N2 ) |~| SKIP
N2 = (a -> N0 [] b -> N3) |~| SKIP
N3 = (a -> N3) [] b -> N3

X0 = SKIP |~| (a -> X0 [] b -> X1)
X1 = SKIP |~| (a -> X0 [] b -> X2)
X2 = a -> STOP [] b -> X3
X3 = SKIP |~| (a -> X0 [] b -> X2)

assert N0 [T= X0
assert N0 [F= X0
assert X0 [T= N0
assert X0 [F= N0
\end{lstlisting}
\begin{description}
\item{\texttt{N0 [T= X0}} Assertion Failed. Gegenbeispiel: Trace <b,b,b,\checkmark> ist in X0, aber nicht in N0 vorhanden. Somit kann X0 kein refinement von N0 sein.\\
\item{\texttt{N0 [F= X0}} Assertion Failed. Gegenbeispiel: Beim trace <b,b,a> ist die Refusal-Menge von X0 $\Sigma\frown\checkmark$, die Refusal-Menge von N0 aber $\Sigma \{a,b\}$.
\item{\texttt{X0 [T= N0}} Assertion Failed. Gegenbeispiel: Der Trace <b,b,a,b> kommt in N0, aber nicht in X0 vor.
\item{\texttt{X0 [F= N0}} Assertion Failed. Gegenbeispiel: Beim Trace <b,b> ist die Refusal-Menge von N0 $\Sigma$, von X0 aber $\Sigma\frown\checkmark \backslash \{a,b\}$.
\end{description}
\end{document}

